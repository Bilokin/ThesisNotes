%%%%%%%%%%%%%%%%%%%%%%%%%%%
%  This is the template for the front and back covers of the thesis 
%  as demanded by the Paris Saclay University. It is a somewhat
%  crude draft, so don't hesitate to adjust it by twitching the
%  \hspace{} and \vspace{} distances in order to meet your needs,
%  and also the font size of the abstract (mine was rather long
%  so I had to invent rather wide fields along with the rather compact text)
%  Also mind that it has to be compiled with LaTeX twice.
%  Enjoy!
%%%%%%%%%%%%%%%%%%%%%%%%%%%

\documentclass[12pt]{article}
\usepackage{textcomp}
\usepackage{tikz}
\usetikzlibrary{calc}
\graphicspath{{images/}}
\usepackage[francais]{babel}
\usepackage{ragged2e}
\usepackage{lipsum}
\newcommand{\ecal}{Si-W ECAL}
\newcommand{\geant}{{\sc geant4}}
\newcommand{\sm}{Standard Model}
\newcommand{\bsm}{Beyond Standard Model}
\newcommand{\gev}{\,GeV}
\newcommand{\ecalp}{\ecal\ physics prototype}
\newcommand{\ttbar}{$t\bar{t}$}
\newcommand{\bbbar}{$b\bar{b}$}
\begin{document}

%%%   define the colour of the title							%%%  
%%%   could be set to match general colour theme 	%%%
\definecolor{SchoolColor}{rgb}{0.145,0.666,1} %%% Cyanish %%

\usetikzlibrary{calc}
\thispagestyle{empty}

%%% the purple border line %%%
\begin{tikzpicture}[remember picture, overlay]
    \draw[line width=1.2 pt, violet!80!red] 
    ($(current page.south west)+(1 cm,+1. cm)$) 
    -- ($(current page.north west)+(1 cm,-1. cm)$) 
    -- ($(current page.north east)+(-1 cm,-1. cm)$) 
    -- ($(current page.south east)+(-1 cm,1. cm)$)
    -- ($(current page.south west)+(1 cm,1. cm)$);
\end{tikzpicture}

{\begin{center}
	\vspace{-3.5cm}
	%%% logo %%%
	\includegraphics[width=14cm]{Logo_ALL.png}\\
	\vspace{1cm}
	
	%%% university title %%%
	\textcolor{violet!80!red!80!black}{{{\uppercase{\Large These de Doctorat de L'Universit\'{e} Paris-Saclay Prepar\'{e}e \`{a} l'Universit\'{e} Paris-Sud}}}}\\
	\vspace{1cm}
	%%% doctoral school title %%%
	\'{E}COLE DOCTORALE N°576\\
	Particules Hadrons \'{E}nergie et Noyau : Instrumentation, Image, Cosmos et Simulation (PHENIICS)\\
	Sp\'{e}cialit\'{e} de doctorat : Physique des particules\par
	\vspace{1.5cm}
	%%% name %%%
 	Par\par  \large \textbf{M. Sviatoslav Bilokin} \par
	\vspace{1cm}
	%%% thesis title %%%
	\Large \textsc{\textcolor{SchoolColor}{
	\textbf{Hadronic showers in a highly granular silicon-tungsten calorimeter and production of bottom and top quarks at the ILC}}}\par
\end{center}

\vspace{1.cm}
\hspace{-1cm}{\textit{Th\`{e}se pr\'{e}sent\'{e}e et soutenue \`{a} Orsay, le 18 juillet 2017} \par}
\vspace{1cm}
\hspace{-1cm}{  Composition de jury: \par}
\hspace{-1cm}{  Prof. Achille Stocchi, Pr\'esident du jury \par}
\hspace{-1cm}{  Dr. Keisuke Fujii, Rapporteur \par}
\hspace{-1cm}{  Dr. Christian Schwahnenberger,  Rapporteur \par}
\hspace{-1cm}{  Dr. Marc Winter, Examinateur \par}
\hspace{-1cm}{  Dr. Gregory Moreau, Examinateur \par}
\hspace{-1cm}{  Dr. Roman P\"oschl, Directeur de th\`{e}se \par}
\hspace{-1cm}{  Dr. Fran\c cois Richard, Membre invit\'{e} \par}
}
\clearpage


% ~~~~~~~~~~~~~~~~~~~~~~~~~~~~~~~~~~~~~~~~
% ~~~~~~~~~~~~~~~~~~~~~~~~~~~~~~~~~~~~~~~~

%%% a lifehuck to adgust the font size and spacing %%%
\makeatletter
\newcommand*\mysize{%
  \@setfontsize\mysize{9.5}{9.0}%
}
\makeatother

\newpage
\thispagestyle{empty}
\begin{tikzpicture}[remember picture, overlay]
	%%% the University+ED logo %%%
    \node [anchor=north west, shift={(1.2 cm,-0.2cm)}] at (current page.north west) {\includegraphics[width=7.5cm]{pheniics.png}};
     \mysize 
    \node [anchor=north, yshift=-2.1 cm, text width=18cm, inner sep=.3cm] (resume) at (current page.north) {
    \begin{minipage}{\linewidth}
    %%% title %%%
\justify{     {\bf Titre:} Un titre long et beau qui prend probablement plus d'une ligne\\
	%%% key words %%%
     			  {\bf Mots cl\'{e}s:} \textit{astrologie, exo-psychologie, arts sombres, Voyage spatial}\\       		
     			  {\bf R\'{e}sum\'{e}:}  
     			  This thesis presents studies for the International Linear Collider (ILC), a linear electron-positron collider with a nominal center-of-mass energy from 250\,GeV to 500\,GeV.
     			  
     			  Data are analysed that were recorded with the physics prototype of the CALICE silicon-tungsten electromagnetic calorimeter (\ecal) prototype at FermiLab in 2008. During this thesis, a track-finding algorithm was developed, which finds secondary tracks in hadronic events recorded by the \ecalp. This algorithm reveals details of hadronic interactions in the detector volume and the results are compared with simulations based on the the \geant\ toolkit.
     			  
     			  Indirect searches of New Physics require a high precision on the measurements of the \sm\ parameters. Theories of physics beyond \sm\ imply modifications of electroweak couplings of the heavy quarks top and bottom. The second part of the thesis is a full simulation study of vertexing algorithms in the ILD environment and the reconstruction of the b-quark charge. The b-quark charge reconstruction is essential for many physics channels at the ILC, particularly, for the $e^+e^-\to b\bar{b}$ and the $e^+e^-\to t\bar{t}$ channels.
     			  The developed algorithm improves the b-quark charge reconstruction performance.
     			  
     			  The b-quark charge reconstruction methods are applied to the \ttbar\ production process. This allows to increase statistics for the top quark electroweak form factor estimation w.r.t an earlier study and thus to decrease corresponding statistical uncertainties.
     			  
     			  The results of the detector study allow for an estimation of the ILC precision on the b-quark electroweak couplings and form factors. The ILC will be able to resolve the LEP anomaly in the \bbbar\ production process. The ILC precision on the right-handed $Z^0b\bar{b}$ coupling, a prime candidate for effects of new physics, is calculated to be at least 5 times better than the LEP experiments. %%% replace by the text of the abstract in French %%%
}
    \end{minipage}
    };
    
    \node [anchor=north, yshift=-0.3 cm, text width=18cm, inner sep=.3cm] (abstract) at (resume.south) { % abstract:
    \begin{minipage}{\linewidth}
    
    %%% title %%%
\justify{     {\bf Title:} A long and beautiful title that probably takes more than one line\\
	%%% key words %%%
     			  {\bf Key words:} \textit{astrology, exo-psychology, dark arts, space travel} \\
    			  {\bf Abstract:}
    			  This thesis presents studies for the International Linear Collider (ILC), a linear electron-positron collider with a nominal center-of-mass energy from 250\,GeV to 500\,GeV.
    			  
    			  Data are analysed that were recorded with the physics prototype of the CALICE silicon-tungsten electromagnetic calorimeter (\ecal) prototype at FermiLab in 2008. During this thesis, a track-finding algorithm was developed, which finds secondary tracks in hadronic events recorded by the \ecalp. This algorithm reveals details of hadronic interactions in the detector volume and the results are compared with simulations based on the the \geant\ toolkit.
    			  
    			  Indirect searches of New Physics require a high precision on the measurements of the \sm\ parameters. Theories of physics beyond \sm\ imply modifications of electroweak couplings of the heavy quarks top and bottom. The second part of the thesis is a full simulation study of vertexing algorithms in the ILD environment and the reconstruction of the b-quark charge. The b-quark charge reconstruction is essential for many physics channels at the ILC, particularly, for the $e^+e^-\to b\bar{b}$ and the $e^+e^-\to t\bar{t}$ channels.
    			  The developed algorithm improves the b-quark charge reconstruction performance.
    			  
    			  The b-quark charge reconstruction methods are applied to the \ttbar\ production process. This allows to increase statistics for the top quark electroweak form factor estimation w.r.t an earlier study and thus to decrease corresponding statistical uncertainties.
    			  
    			  The results of the detector study allow for an estimation of the ILC precision on the b-quark electroweak couplings and form factors. The ILC will be able to resolve the LEP anomaly in the \bbbar\ production process. The ILC precision on the right-handed $Z^0b\bar{b}$ coupling, a prime candidate for effects of new physics, is calculated to be at least 5 times better than the LEP experiments. %%% replace by the text of the abstract in English %%%
}
    \end{minipage}
    };
    
    %%% draw a purple frame around each abstract %%%
    \draw[line width=1 pt, violet!80!red] (resume.south west) -- (resume.north west) -- (resume.north east) -- (resume.south east) -- (resume.south west);
    \draw[line width=1 pt, violet!80!red] (abstract.south west) -- (abstract.north west) -- (abstract.north east) -- (abstract.south east) -- (abstract.south west);
    
    %%% footnote %%%
    \node [anchor=south west, violet!80!red, shift={(1.2 cm,0.5cm)}, inner sep=0.2pt] at (current page.south west) {
    \begin{minipage}{12cm}
    {\bf Universit\'{e} Paris-Saclay} \\
    Espace Technologique / Immeuble Discovery \\
    Route de l'Orme aux Merisiers RD 128 / 91190 Saint-Aubin, France 
    \end{minipage}
    };
    
    %%% the "e" image at the bottom %%%
    \node [anchor=south east, violet!80!red!80!black, shift={(-1.5 cm,0.5cm)}, inner sep=0pt] at (current page.south east) {\includegraphics[width=1.6 cm]{e.png}};
    
\end{tikzpicture}
% ~~~~~~~~~~~~~~~~~~~~~~~~~~~~~~~~~~~~~~~~

\end{document}