\begin{titlepage}

\begin{tikzpicture}[remember picture,overlay,color=blue!20!red!45!black!75!]
	\draw[very thick]
		([yshift=-130pt,xshift=45pt]current page.north west)--     
		([yshift=-130pt,xshift=-35pt]current page.north east)--    
		([yshift=35pt,xshift=-35pt]current page.south east)--      
		([yshift=35pt,xshift=45pt]current page.south west)--cycle; 
\end{tikzpicture}




\thispagestyle{empty}

%% Logos en haut de la page
\begin{textblock}{1}(1.15,0.8)
\includegraphics[height=2.7cm]{graphics/UPSUD.jpg} %% Logo de Paris Saclay
\label{Logo Paris Saclay}
\end{textblock}
\begin{textblock}{1}(12,0)
\includegraphics[height=2.4cm]{graphics/LAL.png}
\label{Logo Etablissement}
\end{textblock}
%\begin{textblock}{13}(1.15,3.3)
%  NNT : 2016SACLS590
%\end{textblock}
%\vspace{3cm}



%\vspace*{-1.0cm}
%\begin{textblock}{1}(1.15,1)
%\includegraphics[height=2.4cm]{logo/UPsac.pdf} %% Logo de Paris Saclay
%\hspace*{6cm}
%\includegraphics[height=2.7cm]{logo/UPS.pdf}
%\label{Logo Paris Saclay}
%\end{textblock}

%\begin{textblock}{13}(1.15,2.6)
%  NNT : 2016SACLS590
%\end{textblock}

\vspace*{2.3cm}
\begin{center} 
\begin{tabular}{c}
		\\
		%\LARGE\textsc{Th�se de doctorat\\ de l'Universit� Paris-Saclay} \\
    {\huge \color{blue!20!red!45!black} \textbf{\textsc{TH\'ESE DE DOCTORAT}}} \\
    \\
    {\LARGE \color{blue!20!red!45!black} \textbf{de l'Universit\`e Paris-Saclay}} \\
    \\
    \LARGE{ \color{blue!20!red!45!black} \textbf{pr\'epar\'ee \'a Universit\'e Paris-Sud}} \\ 
    \\
    \LARGE{ \color{blue!20!red!45!black} \textbf{au sein de l'IRFU/SPP}}\\   \bigskip
		\\
		\\
		%\Large{Ecole doctorale n$^{\circ}\ecodocnum$ 576}\\
		\Large{Particules, Hadrons, \'Energie, Noyau, Instrumentation, Imagerie,}\\
		\Large{Cosmos et Simulation (PHENIICS)}\\
		\\
		\Large{Sp\'ecialit\'e de doctorat: Physique des particules} \\
    \\
		\\
    \\
    \Large{\textbf{par}} 
    \\
    \\
    \LARGE{\textbf{{M. Sviatoslav BILOKIN}}}\\ \bigskip
    \\
		\\
		\Huge\bf{Measurement of the W boson mass }\\
		\\
		\Huge\bf{with the ATLAS detector}\\
    \\
    
		\\
		\\
\end{tabular}


\begin{tabular}{ll}
\Large{Th\'ese pr\'esent\'ee et soutenue \'a Saclay, le 19 Septembre 2016} \\
	  \\
	\\
	\Large\bf\underline{Composition du jury :}\\ \\
	   M. Martin Aleksa     & \hspace{-7cm} Directeur de Recherche, CERN, Examinateur   \\
	   M. Reza Ansari       & \hspace{-7cm} Professeur, Universit\'e Paris-Sud, Pr\'esident du Jury  \\
	   M. Maarten Boonekamp & \hspace{-7cm} Directeur de Recherche, CEA/Irfu, Directeur de th\'ese   \\
	   M. Daniel Froidevaux & \hspace{-7cm} Directeur de Recherche, CERN, Rapporteur   \\
  	   M. Luca Malgeri      & \hspace{-7cm} Senior Scientist, CERN, Rapporteur   \\
           M. Fulvio Piccinini  & \hspace{-7cm} Senior Scientist, Universit\'e de Pavie, Examinateur   \\
    %  M. Boris Tuchming & Directeur de Recherche, CEA/Irfu, Examinateur \\
\end{tabular}

\end{center}
\end{titlepage}

\titlepage
