%\section{Introduction}

The \sm\ of particle physics provides a unified description of electromagnetic, weak and strong forces. It has been developed by a wide scientific community in the middle of 20th century and has been confirmed by a vast majority of experiments. 

The latest triumph of the \sm\ is a confirmation of the Higgs boson existance by Large Hadron Collider (LHC) experiments at CERN, CMS~\cite{bib:HiggsCms} and ATLAS~\cite{bib:HiggsAtlas}, in July 2012.
%TOP?
Another great success of the \sm\ is the discovery of the top quark by TeVatron collaborations in 1995. Top quark is the heaviest elementary particle found, which plays an important role in the \sm\ and cosmology. 
Up to now, the Higgs boson and the top quark were precisely studied in a QCD environment at TeVatron and LHC experiments, while a precise measurement of electroweak coupling constants of the particles is left for the future experiments.

Despite its success, the \sm\ leaves many experimental and theoretical phenomena without a definite answer. Therefore many \bsm (BSM) theories have been developed. 
Since there are many indirect evidences of New Physics, the high energy frontier colliders are required. 

The International Linear Collider \cite{bib:ILC} (ILC) is a high energy electron-positron collider project aimed at precision measurements and New Physics searches. 
The ILC can operate at center-of-mass energy $\sqrt{s}$ of 500\gev, which is ideal for studies of electroweak interactions of the top quark. 
Well known leptonic initial state at the ILC allows clean, model-independent analysis of \sm\ processes as well as for BSM searches. 
%CALO?

The highly granular calorimeters of ILC detectors allow accurate particle separation required by Particle Flow reconstruction algorithms.

This thesis consists of three main parts: theoretical background and ILC description, data analysis of hadronic interactions in a highly granular calorimeter prototype and top quark analysis in the framework of ILD detector model of ILC project with an extension on b quark studies.

Part~\ref{PARTI} gives the necessary backgroud for the thesis subject. A brief introduction into the theoretical framework of the \sm\ provided in Sec.~\ref{sec:SM}, and the description of the ILC project is described in Sec.~\ref{sec:ILC}.

Part~\ref{PARTII} of the thesis concentrates on the analysis of the beam test data recorded with the CALICE Silicon-Tungsten Electromagnetic Calorimeter (\ecal) physics prototype. The granularity of this device allows disentangling fine details of the hadronic interaction events. Sections~\ref{sec:passage}, \ref{sec:ecalintro} provide an introduction to the topic. Section~\ref{sec:track} describes the reconstruction of the secondary tracks from the $\pi^-$ interaction. 
In Sec.~\ref{sec:results} one finds the data-simulation comparison using new variables from the developed track-finding algorithm.

Part~\ref{PARTIII} unites several studies  of the $e^+e^- \to t\bar{t}$ and $e^+e^- \to b\bar{b}$ channels at the ILC.% performed using a full simulation of the ILD experiment. 
Chapter~\ref{sec:Phenomenology} introduces a common theoretical framework of the heavy quark production. Section~\ref{sec:JetChargeReconstruction} describes the methods of the b-jet charge measurement. In Chapters~\ref{sec:TopProduction} and~\ref{sec:BProduction} one finds the application of the b-jet charge technique to the top and bottom quark production studies at the ILC, respectively. The new studies of the expected precision on the bottom quark couplings at the ILC are provided in Sec.~\ref{sec:NewResults}. 











