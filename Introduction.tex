%\section{Introduction}

The \sm\ of particle physics provides a unified description of electromagnetic, weak and strong forces. It has been developed by a wide scientific community in the middle of 20th century and has been confirmed by a vast majority of experiments. 

The latest triumph of the \sm\ is a confirmation of the Higgs boson existance by Large Hadron Collider (LHC) experiments at CERN, CMS~\cite{bib:HiggsCms} and ATLAS~\cite{bib:HiggsAtlas}, in July 2012.
%TOP?
Another great success of the \sm\ is the discovery of the top quark by TeVatron collaborations in 1995. Top quark is the heaviest elementary particle found, which plays an important role in the \sm\ and cosmology. 
Up to now, the Higgs boson and the top quark were precisely studied in a QCD environment at TeVatron and LHC experiments, while a precise measurement of electroweak coupling constants of the particles is left for the future experiments.

Despite its success, the \sm\ leaves many experimental and theoretical phenomena without a definite answer. Therefore many \bsm (BSM) theories have been developed. 
Since there are many indirect evidences of New Physics, the high energy frontier colliders are required. 

The International Linear Collider \cite{bib:ILC} (ILC) is a high energy electron-positron collider project aimed at precision measurements and New Physics searches. 
The ILC can operate at center-of-mass energy $\sqrt{s}$ of 500\gev, which is ideal for studies of electroweak interactions of the top quark. 
Well known leptonic initial state at the ILC allows clean, model-independent analysis of \sm\ processes as well as for BSM searches. 
%CALO?

The highly granular calorimeters of ILC detectors allow accurate particle separation required by Particle Flow reconstruction algorithms.

This thesis consists of three main parts: theoretical background and ILC description, data analysis of hadronic interactions in a highly granular calorimeter prototype and top quark analysis in the framework of ILD detector model of ILC project with an extension on b quark studies.
%This thesis concentrates on top quark analysis in the framework of ILD detector model of ILC project with an extension on b quark studies and data analysis of hadronic interactions in a highly granular calorimeter prototype.

The second part of the thesis concentrates analisys of  beam test data recorded with the CALICE Silicon-Tungsten Electromagnetic Calorimeter (\ecal) physics prototype. The granularity of this device push particle calorimetry...
%DESCRIPTION OF CHAPTERS

\%DESCRIPTION OF CHAPTERS











