\usepackage{wrapfig}
\usepackage{bm}
\usepackage{lineno}
\usepackage{array}
\usepackage{lscape}
\usepackage{graphicx}
\usepackage{subcaption}
\usepackage{float}
\usepackage{multicol} % This is so we can have multiple columns of text side-by-side
\usepackage{amsmath}
\columnsep=100pt % This is the amount of white space between the columns in the poster
\columnseprule=3pt % This is the thickness of the black line between the columns in the poster

\usepackage[svgnames]{xcolor} % Specify colors by their 'svgnames', for a full list of all colors available see here: http://www.latextemplates.com/svgnames-colors

\usepackage{times} % Use the times font
%\usepackage{palatino} % Uncomment to use the Palatino font

\usepackage{graphicx} % Required for including images
\graphicspath{{figures/}} % Location of the graphics files
\usepackage{booktabs} % Top and bottom rules for table
\usepackage[font=small,labelfont=bf]{caption} % Required for specifying captions to tables and figures
\usepackage{amsfonts, amsmath, amsthm, amssymb} % For math fonts, symbols and environments
\usepackage{wrapfig} % Allows wrapping text around tables and figures

\newcommand{\bbbar}{$b\bar{b}$}
\newcommand{\afb}{$A_{FB}^b$}
\newcommand{\sm}{Standard Model}
\newcommand{\bsm}{Beyond Standard Model}
\newcommand{\mF}{\mathcal{F}^I}
