\section*{Conclusions}
This thesis presents new developed methods and studies done for the ILC project.
Detectors at the ILC are designed for the application of Particle Flow algorithms, which enhance the physics performance by using the information from the highly granular calorimeters. 

The high granular \ecalp\ was constructed and tested by the CALICE collaboration. 
During this thesis, a track-finding algorithm was developed allowing to reconstruct secondary tracks emerged from hadronic interactions in the \ecalp.
The \ecalp\ simulation was compared to the data using the new observables from the track-finding algorithm. 
All tested simulation models shows a good performance in terms of the new observables. 
These results have been accepted as CALICE Preliminary results and were presented at the Vienna Conference on Instrumentation (VCI) and the IEEE Nuclear Science Symposium.
The developed track-finding algorithm can be applied for the Particle Flow reconstruction or at least for its validation in simplified scenarios.

In this thesis, the b-quark charge technique is used reconstruction of quark polar angle distribution.
The developed methods are applied in two channels, $e^+e^-\to t\bar{t}$ and $e^+e^-\to b\bar{b}$ processes at $\sqrt{s} = 500$\,GeV and 250\,GeV, respectively.

The developed b-quark charge reconstruction technique uses information from reconstructed secondary and tertiary vertices. 
Using the standard reconstruction algorithm one has a low b-quark charge purity of 66\%.
It was found, that missing particles from the secondary and tertiary vertices degrade the b-quark charge purity. 
%A detailed study was made to establish the reasons behind the missing particles. 
To increase the purity of the b-quark charge, the developed Vertex Charge Recovery algorithm adds the missing particles to the reconstructed vertices and, as a result, it increases the b-quark charge purity by 7\%.
The kaons from the secondary or tertiary vertices deliver information about the initial quark charge. 
To increase the kaon identification efficiency, an angular correction was applied to the reconstructed $dE/dx$ values. 
After correction one has 97\% purity and 87\% efficiency of the kaon identification.

The top quark polar angle is reconstructed in the semileptonic channel using a combination of the b-quark charge with the $W^\pm$ lepton charge. 
This technique improves the reconstruction efficiency by 25\% as comparing to the previous result~\cite{bib:ILCTOP}.
Moreover, the b-quark charge technique can be applied in the fully hadronic \ttbar\ pair decays, which should increase significantly the available statistics to determine the top quark electroweak couplings. 

This thesis addresses the topic of \bbbar\ pair production, which have never been studied using the ILC environment. 
This study is motivated by the $A_{FB}$ measurement at $Z^0$ pole by LEP collaborations, which disagrees by 2.5 standard deviations from the \sm\ prediction.
This deviation shifts the right-handed $Z^0b\bar{b}$ coupling by 25\% with 10\% uncertainty.
The b-quark charge measurement is the only possible method to compute the b-quark polar angle distribution. 
Two major problems were discovered: the event migration and the forward region efficiency decrease. 
It was found, that residual charge impurity contaminates completely the backward region of b-quark polar angle distribution for the left-handed beam configuration.
The procedure of the data-driven charge purity measurement allows to correct the distribution for residual contamination. 
As the result, the reconstructed b-quark polar angle is usable for the extraction of electroweak couplings and form factors of the b-quark. 
Given the accuracy predicted at ILC on the right-handed coupling, 5 times better then at LEP, the LEP anomaly will be either fully confirmed or definitely discarded. 
A paper publication based on this work is on-going.
In view of the observed experimental shortcomings reported in this thesis, the ILD collaboration initiated an optimization of the forward region detectors design to address the forward region inefficiency problem.

%The b-quark charge measurement requires an ultimate performance from all subdetectors and all reconstruction algorithms. 
%The Particle Flow reconstruction scheme is essential for the b-jet charge technique.
