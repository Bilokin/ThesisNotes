

\section*{Conclusions}
This thesis presents new developed methods and studies for the ILC project.
The developed b-jet charge reconstruction technique uses information from reconstructed secondary and tertiary vertices. 
Using the standard reconstruction algorithm one has a low b-jet charge purity of 66\%.
It was found, that missing particles from the secondary and tertiary vertices degrade the b-jet charge purity. 
A detailed study was made to establish the reasons behind the missing particles. 
To increase purity of the b-jet charge, the developed Vertex Charge Recovery algorithm adds the missing particles to the reconstructed vertices and increase the b-jet charge purity by 7\%.
The kaons from the secondary or tertiary vertices have an information about the initial quark charge. 
To increase the kaon identification efficiency, an angular correction was applied to the reconstructed $dE/dx$ values. 
After correction one has 97\% purity and 87\% efficiency of the kaon identification.

In this thesis the b-jet charge technique is used reconstruction of quark polar angle.
The developed methods were applied in two channels, $e^+e^-\to t\bar{t}$ and $e^+e^-\to b\bar{b}$ processes at $\sqrt{s} = 500$\,GeV and 250\,GeV, respectively.
The top quark polar angle was reconstructed in the semileptonic channel using a combination of the b-jet charge and $W^\pm$ lepton charge.
This technique improves the reconstruction efficiency by 25\% as comparing to the previous result~\cite{bib:ILCTOP}.
The b-jet charge measurement requires an ultimate performance from all subdetectors and all reconstruction algorithms. The Particle Flow reconstruction scheme is essential for the b-jet charge technique.

The Particle Flow is possible due to the highly granular calorimeters, which will be equipped at the ILC experiments. 
The \ecalp\ was constructed and tested by the CALICE collaboration. 
During this thesis, a track-finding algorithm was developed allowing to reconstruct secondary tracks emerged from hadronic interactions in the \ecalp.
The \ecalp\ simulation was compared to the data using the new observables from the track-finding algorithm. 
All tested simulation models shows a good performance in terms of the new observables. 
The results were presented at Vienna Conference on Instrumentation (VCI) and IEEE Nuclear Science Symposium.
The developed track-finding algorithm can be applied for the Particle Flow reconstruction. 